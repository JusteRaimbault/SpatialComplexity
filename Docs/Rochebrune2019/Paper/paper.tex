\documentclass[11pt]{article}

% general packages without options
\usepackage{amsmath,amssymb,bbm}
% graphics
\usepackage{graphicx}
% text formatting
\usepackage[document]{ragged2e}
\usepackage{pagecolor,color}

\newcommand{\noun}[1]{\textsc{#1}}

\usepackage[utf8]{inputenc}
\usepackage[T1]{fontenc}
% geometry
\usepackage[margin=1.8cm]{geometry}

\usepackage{multicol}
\usepackage{setspace}



\usepackage{natbib}
\setlength{\bibsep}{0.0pt}

\usepackage[french]{babel}

% layout : use fancyhdr package
%\usepackage{fancyhdr}
%\pagestyle{fancy}

% variable to include comments or not in the compilation ; set to 1 to include
\def \draft {1}


% writing utilities

% comments and responses
%  -> use this comment to ask questions on what other wrote/answer questions with optional arguments (up to 4 answers)
\usepackage{xparse}
\usepackage{ifthen}
\DeclareDocumentCommand{\comment}{m o o o o}
{\ifthenelse{\draft=1}{
    \textcolor{red}{\textbf{C : }#1}
    \IfValueT{#2}{\textcolor{blue}{\textbf{A1 : }#2}}
    \IfValueT{#3}{\textcolor{ForestGreen}{\textbf{A2 : }#3}}
    \IfValueT{#4}{\textcolor{red!50!blue}{\textbf{A3 : }#4}}
    \IfValueT{#5}{\textcolor{Aquamarine}{\textbf{A4 : }#5}}
 }{}
}

% todo
\newcommand{\todo}[1]{
\ifthenelse{\draft=1}{\textcolor{red!50!blue}{\textbf{TODO : \textit{#1}}}}{}
}


\makeatletter


\makeatother


\begin{document}







\title{Espace et complexités des systèmes territoriaux
\\\medskip
\textit{Actes des Journ{\'e}es de Rochebrune 2019}
}
\author{\noun{Juste Raimbault}$^{1,2}$
\\
$^1$ UPS CNRS 3611 ISC-PIF\\
$^2$ UMR CNRS 8504 G{\'e}ographie-cit{\'e}s
}
\date{}


%\pagenumbering{gobble}



\maketitle

\justify

\begin{abstract}
	Le caractère spatialisé des systèmes territoriaux joue un rôle déterminant dans l'émergence de leurs complexités. Cette contribution vise à illustrer dans quelle mesure différents types de complexités peuvent se manifester dans des modèles de tels systèmes.\\\medskip
	\noindent\textbf{Keywords : }\textit{Complexités; Systèmes territoriaux; Morphogenèse urbaine; Co-évolution}
\end{abstract}





\section{Introduction}


Les systèmes territoriaux, qui peuvent être compris comme des structures socio-spatiales auto-organisées, sont une illustration typique de systèmes complexes étudiés sous de nombreux angles incluant plus ou moins l'aspect spatial de ces systèmes. Cette contribution vise à illustrer une approche de ceux-ci par la géographie urbaine, et dans quelle mesure leur complexité est intimement liée à leur caractère spatial.

Notre appréhension des systèmes territoriaux se place plus particulièrement dans la lignée de la théorie évolutive des villes~\citep{pumain1997pour} qui comprend les systèmes urbains comme des systèmes multi-niveaux, dans lesquels la co-évolution des multiples composants et agents détermine la dynamique de ceux-ci~\citep{raimbault2018caracterisation}.

Le concept de complexité correspond à de diverses approches et définitions \citep{chu2008criteria,deffuant2015visions}. Un précédent travail par \cite{raimbault2018relating} s'était proposé de suggérer des ponts épistémologiques entre différentes approches et définitions de la complexité

Nous tâchons ici d'illustrer les liens entre complexité et espace selon différentes vues de celle-ci, à la fois à travers des considérations théoriques mais aussi par l'exploration de modèles de simulation de systèmes territoriaux.




%%%%%%%%%%%%%%%%%
\section{Complexité ontologique}
%%%%%%%%%%%%%%%%%


%%%%%%%%%%%%%%%%%
\subsection{Multidimensionalité et diversité d'approches}


Une première entrée théorique permet une entrée sur ce que nous appelons \emph{complexité ontologique}, qui a été proposée par \cite{pumain2003approche} comme la largeur des points de vue disciplinaires nécessaires pour appréhender un même objet. La multidimensionalité des systèmes territoriaux reste un enjeu principal pour leur compréhension, comme l'illustre \cite{perez2016agent} dans le cas des systèmes multi-agents.


Nous reprenons l'exemple de \citep{raimbault2017invisible} comme preuve-de-concept de la diversité des approches possibles dans le cas des relations entre réseaux et territoires, et suggérons des pistes de réflexion quant au rôle de l'espace dans cette complexité, comme les processus évolutifs de diversification ou de spécialisation liés aux niches spatiales.



%%%%%%%%%%%%%%%%%
\subsection{L'espace porteur de richesse ontologique ?}

À ce point, nous proposons une hypothèse, dont l'exploration empirique nécessiterait des analyses scientométriques poussées hors de portée de ce travail, selon laquelle une spatialisation plus élaborée serait lié à un éventail ontologique plus large.

% exemple new economic geography et graographical economics.
% dans quelle mesure est ce une contingence disciplinaire ? pas dans notre propos puisque on commente des disciplines ici ! (le preciser).
% processus possible : reductuion de l'espace et reduction ontologioque consequences toutes les deux du reductionisme pour resolution analytique ? check raffinement modeles eco. 
% egalement dimension methodo - abm/simu vs analytique/dynsis au plus
% citer chapitre lyons et butterflys





%%%%%%%%%%%%%%%%%
\section{Complexité dynamique}
%%%%%%%%%%%%%%%%%


\subsection{Systèmes dynamiques, chaos et fractales}


La compréhension des systèmes complexes comme systèmes dynamiques aux attracteurs plus ou moins chaotiques a été largement développée en géographie \citep{dauphine1995chaos}.


\subsection{Non-ergodicité et sensibilité aux conditions initiales}

Nous utilisons ici un modèle de morphogenèse urbaine introduit par \cite{raimbault2018calibration} pour illustrer les propriétés de non-ergodicité et de dépendance au chemin des systèmes territoriaux~\citep{pumain2012urban}.

En particulier, nous montrons la forte sensibilité des formes urbaines finales simulées aux perturbations spatiales, et plus généralement la dépendance au chemin des trajectoires pour les indicateurs morphologiques agrégés.



%%%%%%%%%%%%%%%%%
\section{Complexité et co-évolution}
%%%%%%%%%%%%%%%%%


\subsection{Co-évolution dans les systèmes territoriaux}

L'intrication forte des éléments présents au sein de ce qui peut être compris comme niches territoriales, au sens des niches écologiques de \cite{holland2012signals}, est une expression d'une co-évolution et donc d'une complexité au sein de ces niches. \cite{raimbault2018modeling} montre l'existence empirique de ces niches spatiales dans le cas du système de villes français sur le temps long, ainsi que leur émergence au sein d'un modèle de co-évolution entre villes et réseaux de transport à l'échelle macroscopique.


\subsection{Non-stationnarité spatiale et co-évolution}

Nous explorons alors ici par des expériences de simulation le lien entre non-stationnarité spatiale, qui est également un marqueur de complexité spatiale, et émergence de niches au sein d'un modèle de morphogenèse hybride couplant développement urbain et réseau, introduit par~\cite{raimbault2014hybrid}.

Le modèle RBD~\citep{raimbault2014hybrid} couple de manière simple croissance urbaine et évolution du réseau viaire. La flexibilité des régimes qu'il permet de capturer fournit dans~\cite{raimbault2017identification} un test pour une méthode d'identification de causalités spatio-temporelles. Nous étendons ici cette méthode par une détection endogène des zones spatiales sur lesquelles sont estimées les corrélations, afin de montrer l'émergence de niches par la non-stationnarité.

Soit $\mathbf{X}=\vec{x}_{1\leq i\leq N}$ les points générateurs des zones d'estimation (construites par triangulation de Dirichlet). Nous résolvons le problème d'optimisation
\[
\min_{N,\mathbf{X}} f(\tilde{\rho}_i)
\]
où la fonction $f$ donne un objectif en termes d'estimation de la corrélation (par exemple corrélation absolue moyenne $ - 1/N \sum_i \left| \rho_i \right|$, corrélation maximale $ - \max \left| \rho_i \right|$, niveau d'estimation en termes de taille des intervalles de confiance).

En pratique, le problème est résolu de manière heuristique par algorithme génétique, à nombre fixé de centres variant dans $3\leq N \leq 10$ (sachant que le champ non-stationnaire est généré par un nombre fixe de centres $N=6$).

% Experimental setup:
%  - fixed distrib of field centers weight values ; repetitions with varying positions.
%  -> indicators : optimal number of niches ? "modularity" of these ? [idea : construct neighborhood network, weight = correlation, then modularity detection ?)


%%%%%%%%%%%%%%%%
\section{Complexité et émergence}
%%%%%%%%%%%%%%%%



Notre dernière entrée s'intéresse à la complexité en tant qu'émergence faible de structures et autonomie des niveaux supérieurs~\citep{bedau2002downward}, de manière théorique. Nous rappelons le caractère intrinsèquement multi-échelle des systèmes territoriaux, qui se manifeste par exemple dans l'approche des villes comme systèmes au sein de systèmes de villes (\cite{pumain1997pour} enrichissant \cite{berry1964cities}). Par ailleurs, il existe une nécessité actuelle de production de modèles spatiaux l'intégrant effectivement~\citep{rozenblat2018conclusion}, dans le but de modèles effectivement opérationnels.

La difficulté d'endogénéisation de niveau supérieurs autonomes peut par exemple être illustrée par~\cite{lenechet:halshs-01272236} qui propose un modèle de co-évolution entre transport et usage du sol à l'échelle métropolitaine intégrant une structure de gouvernance endogène pour le réseau de transport. Simulant les négociations entre acteurs locaux du transport, certains régimes conduisent à l'émergence d'un niveau intermédiaire de gouvernance issu de la collaboration entre acteurs voisins.

Des systèmes de complexité ontologique moindre présentent aussi un rôle crucial de l'émergence faible. L'état local d'un flux de trafic est en partie conséquence de l'état global du système.




\section{Discussion}


\subsection{Perspectives}

Diverses approches de la complexité que nous n'avons pu aborder sont suggérées en perspective, comme le lien entre complexité computationnelle et caractère de grande déviation des configurations territoriales (impossibilité empirique en probabilité de les obtenir en force brute), ou le lien entre complexité informationnelle et diffusion de l'innovation dans les systèmes territoriaux \citep{favaro2011gibrat}.



Nous suggérons finalement, suivant \cite{raimbault2017complexity}, que les systèmes territoriaux sont nécessairement au croisement de multiples complexités, et ajoutons, d'après les divers exemples développés ici, que leur caractère spatial prend une place importante dans l'émergence de celles-ci.



%%%%%%%%%%%%%%%%%%%%
%% Biblio
%%%%%%%%%%%%%%%%%%%%
%\tiny

%\begin{multicols}{2}

%\setstretch{0.3}
%\setlength{\parskip}{-0.4em}


\bibliographystyle{apalike}
\bibliography{biblio}
%\end{multicols}



\end{document}
